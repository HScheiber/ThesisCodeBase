% "spell_check": true,
% "dictionary": "Packages/Language - English/en_US.dic"
\documentclass[preprint,aps,prb,floatfix]{revtex4-1}
%\documentclass[journal=jpcbfk, manuscript=article]{achemso}
%\usepackage{achemso}
%\setkeys{acs}{usetitle = true}
%\usepackage[pdftex]{graphicx,color}
\usepackage{xcolor}
\usepackage{graphicx}

\usepackage{amsfonts}
\usepackage{amssymb}
\usepackage{bm}
\usepackage[figuresleft]{rotating}
\usepackage[mathscr]{eucal}
\usepackage{threeparttable}
\usepackage{amsmath}
\usepackage[sort&compress]{natbib}
\usepackage{gensymb}

\usepackage{multirow}
\usepackage{booktabs}

% End of Section Numbering Customization

\AtBeginDocument{
\heavyrulewidth=.08em
\lightrulewidth=.05em
\cmidrulewidth=.03em
\belowrulesep=.65ex
\belowbottomsep=0pt
\aboverulesep=.4ex
\abovetopsep=0pt
\cmidrulesep=\doublerulesep
\cmidrulekern=.5em
\defaultaddspace=.5em
}

\date{\today}

\begin{document}
\author{A. Soni}
\author{G. N. Patey\footnote{Electronic mail: patey@chem.ubc.ca}}
%\email{patey@chem.ubc.ca}
\affiliation{Department of Chemistry, University of British Columbia,
Vancouver, British Columbia, Canada V6T 1Z1}

\title{Simulations of water structure and the possibility of ice nucleation on selected crystal planes of K-feldspar}

\newcommand{\kJmol}{kJ mol$^{-1}$}
\newcommand{\boldr}{{\bm r}}

\begin{abstract}

Molecular dynamics simulations are employed to investigate the structure of supercooled water (230 K) in contact with the (001), (010), and (100) surfaces of potassium feldspar (K-feldspar) in the microcline phase. Experimentally, K-feldspar and other feldspar minerals are known to be good ice-nucleating agents, which play a significant role in atmospheric science. Therefore, a principal purpose of this work is to evaluate the possibility that the K-feldspar surfaces considered could serve as likely sites for ice nucleation. The (001) and (010) surfaces were selected for study because they are perfect cleavage planes of feldspar, with (001) also being an easy cleavage plane. The (100) surface is considered because some experiments have suggested that it is involved in ice nucleation. Feldspar is modeled with the widely used CLAYFF force field, and the TIP4P/Ice model is employed for water. We do not observe ice nucleation on any of the K-feldspar surfaces considered, moreover, the density profiles and the structure of water near these surfaces do not exhibit any particularly ice-like features. Our simulations indicate that these surfaces of K-feldspar are likely not responsible for its excellent ice nucleating ability. This suggests that one must look elsewhere, possibly at water-induced surface rearrangements, or some other ``defect'' structure for an explanation of ice nucleation by K-feldspar.  

\end{abstract}

\maketitle

\section{Introduction}

Recently it has become increasingly clear that feldspar minerals function as very important ice nuclei in the Earth's atmosphere.\cite{AMWW13,YLA13,VMWO17,AWKE14,ZWSH08,HWCH16,PKHE16,KMHP16,ZBHB15,WHKK17,HWTO19,KBPC17} There are a variety of feldspar types differing in composition and crystal structure, but experimental evidence\cite{HWCH16,PKHE16,KMHP16,ZBHB15,WHKK17} indicates that alkali feldspars, containing Na$^{+}$ and/or K$^{+}$ ions, are the most efficient ice nucleating feldspars, with potassium feldspar (K-feldspar) being the best overall.\cite{HWCH16,KMHP16,ZBHB15} It has also been shown that feldspars act as efficient ice nuclei in both deposition\cite{YLA13,ZWSH08,KBPC17} and immersion\cite{AMWW13,AWKE14,HWCH16,PKHE16,KMHP16,ZBHB15,WHKK17,HWTO19}freezing modes. 

Despite the importance of feldspar as an atmospheric ice nucleating agent, the mechanism, or possibly mechanisms, of feldspar ice nucleation remain unclear. However, several possible mechanisms have been suggested based on experimental studies. Exploiting a range of experimental measurements, and comparing different feldspars and other minerals, Zolles {\it et al.}\cite{ZBHB15} suggested that the hydration of counterions near the surface could explain differences in the ice nucleating ability of different feldspars. Their arguments provide a rationalization for the high ice nucleating activity of K-feldspar, relative to other feldspar minerals. Kiselev {\it et al.}\cite{KBPC17} have recently reported a deposition freezing investigation of relatively well characterized K-feldspar surfaces. These experiments led to the conclusion that ice nucleation occurs on active sites, with the exposure of the (100) plane to water being a possible key factor in ice nucleation. This conclusion was supported by calculations of minimum energy ice structures on various feldspar surfaces, which showed that the (100) surface achieved the lowest energy, suggesting a good match with ice. Finally, employing immersion freezing experiments, Whale {\it et al.}\cite{WHKK17} have shown that in their particular samples ice nucleation tends to occur near boundaries between sodium-rich and potassium-rich feldspar phases. They suggest that ice nucleation is somehow related to the ``microtexture'' in these regions. 

During the past few years, computer simulations have begun to make important contributions to the microscopic understanding of ice nucleation,\cite{SCCFPZM16} including both homogeneous\cite{MM10,ESVV14,HD15} and heterogeneous\cite{LHV14,QOHM17,CKSM15,ZBP15a,FD14,CRKSM14,ZBP15b} cases. Molecular dynamics (MD) simulations have suggested possible ice nucleation mechanisms for silver iodide,\cite{ZBP15a,FD14} kaolinite,\cite{CRKSM14,ZBP15b} as well as other surfaces.\cite{LHV14,QOHM17,CKSM15} To our knowledge, the only attempts to observe ice nucleation on any feldspar surface employing MD simulations were carried out by Zielke\cite{Z16} for the (001) and (010) planes of the orthoclase phase of K-feldspar. Density functional theory (DFT) has been used together with guided structure methods to identify low energy structures of water monolayers and bilayers on the (001) surface of K-feldspar in the microcline phase.\cite{PCSM16} Also, as mentioned above, similar calculations have been employed to investigate ice-feldspar interactions for several feldspar surfaces.\cite{KBPC17}

In the present paper we report MD simulations of supercooled bulk water in contact with different K-feldspar (microcline) surfaces, including the perfect cleavage planes, (001) and (010), as well as three variations of (100), which has been suggested as a possible nucleation site.\cite{KBPC17} Additionally, we attempt to model a ``defect'' consisting of both (001) and (100) surfaces, similar to the description of the ice nucleating site given in Ref. \onlinecite{KBPC17}. We do not observe ice nucleation for any of these systems. Furthermore, close examination of water density profiles near these surfaces, and of the surface water layers, do not reveal any particularly ice-like structures. Our results suggest that the ice nucleating ability of K-feldspar is due to some other crystal plane, or, perhaps more likely, to some chemical rearrangement or modification of the water-exposed surfaces.  

The remainder of this paper is divided into three Sections. The models and methods are described in Section \ref{modmed}, results are discussed in Section \ref{results}, and our main observations and conclusions are summarized in Section \ref{summary}.

\section{Models and Methods}
\label{modmed}

We consider five different surfaces of the microcline phase of potassium feldspar (KAlSi$_{3}$O$_{8}$). These include (001) and (010) surfaces, as well as three versions of the (100) plane, which we refer to as (100)a, (100)b, and (100)c. As noted above, (001) and (010) are perfect cleavage planes of feldspar, and a recent experimental and computational study has  attributed ice nucleation to the (100) surface.\cite{KBPC17} Unless otherwise noted, in the present study, the feldspar structure is assumed to be fully ordered with all Al atoms occupying T1(o) lattice sites.\cite{KR87} For all surfaces, except (100)c, the unit cell parameters employed ($a = 0.8592$ nm, $b = 1.2963$ nm, $c = 0.7222$ nm, $\alpha = 90.62^{\circ}$, $\beta = 115.95^{\circ}$, $\gamma = 87.67^{\circ}$) are values based on experimental measurement.\cite{KR87} The (100)c surface is structurally analogous to (100)a, except that its lattice parameters are somewhat different, as described below. 

There are at least two factors that determine whether or not a particular crystal might serve as a good ice nucleus.  One is a good lattice match with ice, and the other relates to details of the microscopic surface structure. The former can be calculated based merely on the lattice parameters,\cite{PK10} whereas the latter requires detailed analysis such as can be provided by molecular simulations.\cite{LHV14,QOHM17,CKSM15,ZBP15a,FD14,CRKSM14,ZBP15b} The percentage lattice mismatch is defined as\cite{PK10} 

\begin{equation}
\delta = \frac{na_{n} - ma_{i}}{ma_{i}} \times 100\%\ ,
\label{eq:mismatch}
\end{equation}
where, $a_{n}$ and $a_{i}$ represent lattice parameters of the potential ice nucleus and of ice, respectively, and $n$ and $m$ are integers chosen to minimize $\delta$. The percentage lattice mismatches relevant for the (001), (010), (100)a, and (100)b feldspar surfaces are given in Table \ref{tab:latticematch}. We note that both (100) surfaces (4.5\%,1.2\%), and the (010) surface (4.1\%,1.2\%) give reasonable matches to the prism face of ice I$_{h}$, whereas, the (001) surface (4.1\%,4.5\%) best matches the basal plane of I$_{h}$. There are no good matches to cubic ice, I$_{c}$. 

Top views of the (001), (010), (100)a, and (100)b surfaces are given in Fig. \ref{fig:topview}. The (100)c surface closely resembles (100)a and is not shown. In all cases, protons have been ``attached'' to all surface oxygen atoms with dangling bonds using a fixed O-H bond length of 0.1 nm, consistent with the CLAYFF force field.\cite{CLK04} Surfaces (100)a and (100)b correspond to two possible cleavages along the (100) plane. For (100)b all surface protons lie in the same plane, whereas for (100)a half of the surface protons lie in an outer plane, and half in an inner plane closer to the surface. We note that the surface proton structure of (100)a is consistent with that of (100)c.\cite{KBPC17} Our motivation for considering (100)b is that it more closely resembles the ice-nucleating, Al-surface of kaolinite, where all surface protons lie in the same plane.\cite{ZBP15b}

Except for some simulations with (100)c (discussed in Section \ref{scipap}), the surface protons are free to move during the simulation such that the O-H bonds can change direction. We learned from earlier work with the Al-surface of kaolinite,\cite{ZBP15b} that allowing surface O-H bonds to rotate can be  crucial for ice nucleation. Apart from the motion of these surface protons, all other atoms in the feldspar lattice are held fixed, unless otherwise stated. The TIP4P/Ice water model,\cite{ASFV05} which has a normal freezing point of $270 \pm 3$ K,\cite{FAV06} is used throughout. The surface-water interactions are calculated using CLAYFF parameters,\cite{CLK04} except for (100)c, where some modifications were used as described below. Note that the CLAYFF partial charges were in some cases slightly modified in order to give electrically neutral finite slabs. The explicit parameters used in all simulations are given in the Supplementary Material (Tables S1 and S2). 

The (100)c surface was given in Ref. \onlinecite{KBPC17}. The lattice parameters for this surface were obtained with the aid of DFT calculations. The coordinates of the atoms for the (100)c surface  are given explicitly in the Supplementary Material of Ref. \onlinecite{KBPC17}, together with the coordinates of minimum energy ice structures in contact with this surface. We replicated the 6 unit cell structure given in Ref. \onlinecite{KBPC17} (together with the associated ice layers, where applicable) to obtain the coordinates of a slab consisting of 24 unit cells in the XY plane. These coordinates are used in the MD simulations with (100)c as discussed in Section \ref{results}. We also note that the $\sigma$ values given in Ref. \onlinecite{KBPC17} (Supplementary Material) are smaller than those used in the original CLAYFF, presumably to account for the change in lattice parameters. In our simulations of (100)c, we use the values given in Ref. \onlinecite{KBPC17}, and included in Table S2. 

The experiments reported in Ref. \onlinecite{KBPC17} suggest that ice nucleation occurs at a particular defect in the (001) surface, which exposes a portion of the (100) surface to water. This led the authors to suggest that the (100) surface serves as an ice nucleus. Therefore, we also attempt to simulate the defect described in Ref. \onlinecite{KBPC17}, where water is in contact with both (001) and (100)a surfaces. The explicit structure of this model ``defect'' is given in Section \ref{scipap}.    

All simulations employed the GROMACS version 5.1.4 (double precision) software package.\cite{AMSS15} The simulation cells consisted of two mirrored slabs as in earlier work\cite{ZBP15a,ZBP15b} (see Fig. \ref{fig:finalsnapshots}, below) with vacuum spaces (1.0 nm thick, measured from the unit cell boundary of the slab to the edge of the simulation cell) at each end to reduce unwanted interactions resulting from slab images in the direction perpendicular to the slabs. Simulations were carried out under NVT conditions, with the temperature controlled using the Nos\'{e}-Hoover thermostat.\cite{N84,H85} The timestep used in all simulations was 2 fs. The short-range interactions were truncated with a spherical cutoff of 0.9 nm, and the electrostatic interactions were calculated using the particle mesh Ewald method.\cite{DYP93,EPBD95} Molecular bonds were constrained employing the LINCS algorithm,\cite{HBBF97} and ice-like structures were identified using the CHILL algorithm.\cite{MLWSM10}

Details of the systems simulated are given in Table \ref{tab:simsummary}. For each system, trial simulations were performed with different initial conditions as indicated in the table. All systems were first run for 1 ns at 300 K, and then cooled to 230 K employing a simulated annealing process over 1 ns. Final configurations obtained from the cooling process were used to initiate ``production'' runs at 230 K. The bulk water densities given in Table \ref{tab:simsummary} are those at  the center of the simulation cell, where the density profiles are smooth and uninfluenced by surface structure. Note that consistent with earlier simulation studies of ice nucleation,\cite{ZBP15a,ZBP15b} we quote bulk water densities at 300 K, rather than at supercooled temperatures. The reason for this is that at 300 K the liquid density is a reproducible property of the system, whereas in NVT simulations at supercooled temperatures, the liquid water density will vary depending on whether or not ice has nucleated, and if nucleation has occurred, on how much ice has grown in the simulation cell.

From Table \ref{tab:simsummary} we see that, as in previous simulations of ice nucleation,\cite{ZBP15a,ZBP15b} the densities used in NVT simulations are a little less than that of bulk water. This is done to accommodate the larger molecular volume (lower density) of ice compared to liquid water. In these systems the initial pressure is usually negative, but increases becoming positive as ice grows in the system. This continues until the density of the remaining liquid water is too high for further ice growth. Ideally, one would prefer to carry out NPT simulations at 1 bar, but with atomistic slab geometry this is not feasible with GROMACS code. However, one way to test that the NVT conditions are not seriously influencing our results is to carry out simulations where water is in contact with only one slab, and an ``empty'' space is left between the water surface and the slab at the other end of the simulation cell (see figures in the Supplementary Material for an illustration). In principle, this system would equilibrate to the equilibrium vapor pressure of water at the selected temperature, but, due to time restrictions on simulations, the actual conditions likely correspond more closely to zero pressure. In any case, this simulation setup allows ice to nucleate (or not) at essentially constant pressure. 

In previous work,\cite{ZBP15b} we found that in such constant pressure simulations ice nucleation occurred just as in NVT simulations at a lower density. Nevertheless, here we performed such simulations on three surfaces, namely, (001), (100)c, and (100)c with two fixed ice layers. Simulation details and the results obtained are given in the Supplementary Material. The behavior observed is just as described above for the NVT simulations. In particular, ice does not nucleate on the (001) and (100)c surfaces, but does nucleate and grow for (100)c with two fixed ice layers. Also we note that the bulk liquid water density that we obtain in these simulations at 272.2 K ($\sim 0.98$ g/cc) is in good agreement with the value (0.986 g/cc) reported\cite{ASFV05} for TIP4P/Ice at 272.2 K and 1 bar. 


\section{Results and Discussion}
\label{results}

\subsection{The (001), (010), (100)a, and (100)b surfaces}
\label{expsurpar}

We state at the outset that, on the time scales of our simulations, ice nucleation was not observed on the (001), (010), (100)a, and (100)b surfaces of microcline. This also applies to the (100)c surface discussed in detail below.
We note that Zielke\cite{Z16} also did not find ice nucleation in his simulations involving the (001) and (010) surfaces of orthoclase. 

We checked for ice-like structures using the CHILL algorithm,\cite{MLWSM10} as in early work with kaolinite\cite{ZBP15b} and silver iodide\cite{ZBP15a} surfaces, for which ice nucleation and growth at 230 K were easily detected on time scales less than 200 ns. The numbers of molecules in ice-like structures, as identified by CHILL, are plotted versus time in Fig. \ref{fig:chill}. We see that, while the CHILL algorithm does report a few ``ice'' molecules, these just result from fluctuations in supercooled water, and ice nucleation is not detected for these systems. Compare, for example, the present results with similar simulation plots for the Al-surface of kaolinite (e.g., Fig. 1 of Ref. \onlinecite{ZBP15b}), where we see the number of ice molecules grow very rapidly to several thousand once ice nucleation has occurred.

Configurational snapshots at the end of typical simulation runs are shown in Fig. \ref{fig:finalsnapshots}. These snapshots show no hint of ice-like structure, which is easily observable in systems where ice growth has occurred (e.g., see Fig. \ref{fig:2rigidicelayers}, Section \ref{scipap}, below). It is important to note that experimentally feldspar appears to be a significantly  better ice nucleating agent than kaolinite.\cite{AMWW13,AWKE14,ZWSH08} Therefore, we would expect to observe ice nucleation on faster, or at least equivalent time scales, if the surfaces we have considered were responsible for feldspar ice nucleation. 

It is of interest to examine the structure of water on these feldspar surfaces. The number density profiles for the hydrogen and oxygen atoms of water together with those of the mobile surface hydroxyl hydrogen atoms are shown in Fig. \ref{fig:densities}. Corresponding snapshots of the water layer on the surfaces are shown in Fig. \ref{fig:surfacestructure}. From Fig. \ref{fig:densities} we see that, apart from minor oscillations, the water density profiles show structural features only in the vicinity of the surfaces. Consistent with the CHILL results, there are no signs of ice-like structure in these density profiles. It is instructive to compare the density profiles for the (001) feldspar surface with those for the corresponding surface (the Al-surface) of kaolinite, which readily nucleates ice. Comparing present results with those of kaolinite (Fig. 4, Panel C, of Ref. \onlinecite{ZBP15b}), we see some similarities but, importantly, for the feldspar surface the water hydrogen profile has an initial peak with a small inner sub-peak, rather than a pair of distinct peaks as observed for kaolinite. The pair of peaks results from an ice-like bilayer structure at the kaolinite surface, which leads to ice nucleation. We note also that, unlike the kaolinite case, snapshots of water on the (001) surface (Fig. \ref{fig:surfacestructure}) do not show hexagonal rings characteristic of ice-like structure (see Fig. 4, Panels C and D, of Ref. \onlinecite{ZBP15b}). Water at the (010) surface of feldspar exhibits some structural features, but nothing that hints of ice-like structure. The (100)a and (100)b surfaces show two near-surface peaks in the water hydrogen profiles, but the water structures on these surfaces (Fig. \ref{fig:surfacestructure}), while exhibiting some order, do not appear to be sufficiently ice-like for nucleation to occur.

We also carried out a few ``test'' simulations with particular surface variations, which did not lead to ice nucleation, but are worth mentioning. In bulk microcline, the Al atoms prefer to occupy T1(o) sites,\cite{KR87} and this is the structure used in our simulations. However, DFT calculations\cite{PCSM16} suggest that for the (001) surface layer, the Al atoms actually prefer T2 sites. The main difference is that for T1(o) occupation, the outer surface has alternating Al-OH and Si-OH groups exposed to water, whereas, with T2 occupation all exposed hydroxyl groups are Si-OH. We simulated a (001) surface with all Al atoms in T2 positions. The results obtained were similar to those found for T1(o) occupation, and did not show promise of ice nucleation. Some simulations were performed with partial and full deprotonation of the surface hydroxyl groups, and of systems where the potassium ions were allowed to move, such that some entered the solution. Different temperatures were also tried. However, none of these variations led to ice nucleation, nor to ice-like structures near the surface.   

\subsection{The (100)c surface, and a model ``defect''}
\label{scipap}

In Ref. \onlinecite{KBPC17}, a (100) microcline surface with lattice parameters refined by means of DFT calculations\cite{PCSM16} was considered. As stated above, we refer to this surface as (100)c. A top view of this surface (not shown) is very similar to the view of (100)a given in Fig. \ref{fig:topview}. In Ref. \onlinecite{KBPC17}, minimum energy calculations were reported for two and four layers of hexagonal ice on (100)c, and other feldspar surfaces. These calculations determined that, among the surfaces considered, ice interacted most favorably with (100)c. These calculations were used to support experimental observations suggesting that exposed portions of the (100) feldspar surface serve as active sites for ice nucleation. However, no MD simulations of supercooled water in contact with the (100)c surface at finite temperatures were reported, and the stabilities of the ice structures in contact with the (100)c surface were not examined at finite temperature. 

Here we report temperature-dependent MD simulations, carried out to explore the ice nucleating ability of the (100)c surface, and the stability of ice structures in contact with this surface. In our simulations we employ the TIP4P/Ice water model, rather than the SPC water model\cite{BPGH81} used in Ref. \onlinecite{KBPC17}. The SPC model is not a good choice for simulation studies of ice nucleation because it has been shown\cite{VSA05} that for this model Ice II, rather than Ice I$_{h}$, is the stable solid at 1 bar. For testing purposes, we did carry out a few MD simulations with the SPC model, but did not observe ice growth, even when Ice I$_{h}$ surfaces were included as part of the initial configuration. This was true for temperatures as low as 190 K, which is below the reported normal freezing point (to Ice II) of the SPC model.\cite{VSA05} Hence, we report results only for the TIP4P/Ice model. All simulations were initiated using the feldspar (and where appropriate ice) coordinates given in the Supplementary Material of Ref. \onlinecite{KBPC17}. Also, the modified CLAYFF parameters given in the same reference and noted above (see Tables S1 and S2) were employed.

Initially, we performed direct MD simulations at 230 K (as for the other feldspar surfaces) in an attempt to directly observe ice nucleation on (100)c. Simulations were performed both holding the feldspar surface protons fixed in the minimum energy, ice-favorable positions reported in Ref. \onlinecite{KBPC17}, and allowing the protons to move as for the other surfaces described above. Ice nucleation was not observed in any simulation, and we obtained water density profiles  which were very similar to the results for the (100)a surface shown in Fig. \ref{fig:densities}. Since ice nucleation was not observed, we carried out a number of additional simulations to further examine the stability of ice-like structures on (100)c. 

In one set of simulations, we simply ran MD simulations (up to 150 ns) at finite temperatures for the two and four ice layer structures reported in Ref. \onlinecite{KBPC17}. All atoms of the feldspar lattice, including the surface protons, were held fixed during these simulations. Snapshots showing final configurations for the two-layer ice structures at temperatures of 50, 100, and 150 K are shown in Fig. \ref{fig:2icelayers}. We note that at 50 K, two ice layers appear stable and show no signs of disordering. This is qualitatively consistent with the energy minimization results (corresponding to 0 K) reported in Ref. \onlinecite{KBPC17}. However, at and above 100 K, the ice structure is clearly unstable and disordering is obvious in the snapshots. At 230 K disordering is very rapid and the ice structure disappears in less than 10 ns. Similar observations (figures not shown) were made for four ice layers, which appeared stable up to 150 K, but disordered at 200 K. Results for four ice layers at 230 K are shown in in Fig. \ref{fig:4icelayers}. We note that in this case some disordering is apparent at 2 ns, and the system is completely disordered at 10 ns. Thus, finite ice layers are only stable on the (100)c surface at temperatures much below the freezing point of water.

Given these observations, we would not expect (100)c to act as a good ice nucleus under atmospheric conditions,\cite{AMWW13} nor at the temperatures normally employed in experiments.\cite{KBPC17,WHKK17} Nevertheless, to further investigate the ice-(100)c interaction we carried out additional simulations at 230 K. All feldspar atoms were held fixed in their minimum energy positions during these simulations. Two sets of results illustrated by snapshots are shown in Figs. \ref{fig:2rigidicelayers} and \ref{fig:2flexibleicelayers}. In these simulations two ice layers (in the minimum energy configuration) were initially placed in the system together with liquid water, and the simulations were allowed to run. In the first case (Fig. \ref{fig:2rigidicelayers}), the two ice layers were held rigid during the entire simulation, and, as one would expect, ice quickly ``nucleates'' on the rigid ice surface and grows to fill the simulation cell. However, it is interesting to note that there does not appear to be a good ``match'' between ice and the upper feldspar surface. In the second case (Fig. \ref{fig:2flexibleicelayers}), the two ice layers are not held rigid (are flexible) and all water molecules are allowed to move during the simulation. In this case, some ice-like order initially appears at the ice-liquid interface, but nucleation is not achieved, and all ice present eventually melts. 

Analogous simulations were carried out with four ice layers initially present. If the four ice layers are held rigid, ice nucleates and grows to fill the cell (figure not shown), just as is observed for two ice layers. If the initial ice layers are not held rigid, the situation differs markedly from the two layer case. The four flexible ice layers remain stable sufficiently long for ice to nucleate and grow to fill the simulation cell, as  illustrated in Fig. \ref{fig:4flexibleicelayers}. However, we note that the ice near the surface becomes somewhat disordered, and does not maintain the initial contact structure with the surface. Thus, it appears that even when there is a large amount of bulk ice in the system, the ice-surface match is poor at 230 K. 

To further test this observation, we took the final configuration of the two rigid ice layers simulation, and ran it further  allowing the ice molecules, previously held rigid, to move. Initial and final snapshots of this simulation are shown in Fig. \ref{fig:snapshotsinitialfinal}, and we see that even with an initially full cell of bulk ice, the surface layer disorders. Initial and final density profiles for the water oxygen atoms are plotted in Fig. \ref{fig:oxygendensity}, and the disordering at the surface is apparent. For comparison, the liquid water density profile obtained for (100)c at 230 K is included in Fig. \ref{fig:oxygendensity}, and we see that near the surface the profile is similar for both frozen and liquid systems. It is interesting to note that this behavior appears to be consistent with a discussion of ice premelting recently given by Qiu and Molinero.\cite{QM18} These authors argue that if heterogeneous ice nucleation is thermodynamically unfavorable for a particular surface, then one would expect to observe a premelted liquid-like layer in contact with the surface. Furthermore, this layer could appear as a continuous transition at temperatures well below the melting point of ice, which is in qualitative accord with our observation. 

As mentioned above, we also attempted to model a ``defect'' similar to that associated with ice nucleation in Ref. \onlinecite{KBPC17}. At this defect water is in contact with both (001) and (100)a surfaces, as can be seen in Fig. \ref{fig:defectsnapshot}. Fig. \ref{fig:defectsnapshot} shows the final configuration of water after a simulation run at 230 K, and no trace of ice-like structure can be seen. This is consistent with the CHILL algorithm plot included in Fig. \ref{fig:chill}. 
 
\section{Summary and Conclusions}
\label{summary}

In this paper we report MD simulations of supercooled water in contact with the (001), (010), and three versions of the (100) surface of K-feldspar in the microcline phase. We do not observe ice nucleation on any of these surfaces. We  
present a detailed examination of the water density profiles, and related structure, near these surfaces, and while significant surface-induced ordering of the water is observed, we find nothing that closely resembles ice. Thus, these surfaces, at least in their unaltered bulk crystal structure, do not appear to be responsible for the excellent ice nucleating ability of feldspar. 

We pay particular attention to the (100) surface because recent experiments and energy minimization calculations have suggested that this surface is responsible for ice nucleation.\cite{KBPC17} More specifically, in these experiments ice nucleates at defects in the (001) surface, which are believed to expose areas of the (100) surface to water. 
However, our simulations do not support this conclusion. We find that the minimum energy ice structure at the (100) surface reported in Ref. \onlinecite{KBPC17} is stable only at very low temperatures, and at 230 K the water structure near the surface is disordered, even when the remainder of the simulation cell is filled with bulk ice. Additionally, we performed simulations for a defect in (001) that exposes (100) similar to that described in Ref. \onlinecite{KBPC17}, but found no sign of ice-like structure. Of course, our simulations are for a rigid model surface, and it is possible that some structural rearrangement or perturbation of the (100) surface favors ice nucleation in real feldspar. 

To summarize, our results suggest that the surfaces we have considered are unlikely candidates for feldspar ice nucleation sites. This may not be surprising because there are experimental\cite{FTGH00,FCPZ03} and simulation\cite{KLI08} studies which indicate that feldspar (orthoclase) surfaces can undergo ion exchange and possibly structural changes when in contact with water. More recent experiments\cite{WHKK17,HWTO19} point to microtexture, possibly associated with phase boundaries in feldspar samples, as favorable sites for ice nucleation. These observations, taken together with the present simulation results, suggest that ice nucleation occurs on some structurally rearranged or chemically altered surface, rather than on a crystal plane of bulk feldspar.   

\section*{Supplementary Material}
See supplementary material for tables of the force field parameters used in our simulations. Results from constant pressure simulations are also given. 

\begin{acknowledgments}
The financial support of the Natural Science and Engineering Research Council of Canada is gratefully acknowledged. This research has been enabled by the use of WestGrid and Compute/Calcul Canada computing resources, which are funded in part by the Canada Foundation for Innovation, Alberta Innovation and Science, BC Advanced Education, and the participating research institutions. WestGrid and Compute/Calcul Canada equipment is provided by IBM, Hewlett Packard and SGI.
\end{acknowledgments}

\clearpage

\section*{References}
\begin{thebibliography}{10}

%papers on the role of feldspar in nucleation

%importance in the atmosphere
\bibitem{AMWW13} J. D. Atkinson, B. J. Murray, M. T. Woodhouse, T. F. Whale, K. J. Baustian, K. S. Carslaw, S. Dobbie, D. O'Sullivan, and T. Malkin, Nature, {\bf 498}, 355 (2013). 

\bibitem{YLA13} J. D. Yakobi-Hancock, L. A. Ladino, and J. P. D. Abbatt, Atmos. Chem. Phys. {\bf 13}, 11175 (2013).

\bibitem{VMWO17} J. Vergara-Temprado, B. J. Murray, T. W. Wilson, D. O'Sullivan, J. Browse, K. J. Pringle, K. Ardon-Dryer, A. K. Bertram, S. M. Burrows, D. Ceburnis, P. J. DeMott, R. H. Mason, C. D. O'Dowd, M. Rinaldi, and K. S. Carslaw, Atmos. Chem. Phys. {\bf 17}, 3637 (2017). 

\bibitem{AWKE14} S. Augustin-Bauditz, H. Wex, S. Kanter, M. Ebert, D. Niedermeier, F. Stolz, A. Prager, and F. Stratmann, Geophys. Res. Lett. {\bf 41}, 7375 (2014).  

\bibitem{ZWSH08} F. Zimmerman, S. Weinbruch, L. Sch\"{u}tz, H. Hofmann, M. Ebert, K. Kandler, and A. Worringen, J. Geophys. Res. {\bf 113}, D23204 (2008). 

%compare different feldspars

\bibitem{PKHE16} A. Peckhaus, A. Kiselev, T. Hiron, M. Ebert, and T. Leisner, Atmos. Chem. Phys. {\bf 16}, 11477 (2016).

\bibitem{HWCH16} A. D. Harrison, T. F. Whale, M. A. Carpenter, M. A. Holden, L. Neve, D. O'Sullivan, J. V. Temprado, and B. J. Murray, Atmos. Chem. Phys. {\bf 16}, 10927 (2016). 

\bibitem{KMHP16} L. Kaufmann, C. Marcolli, J. Hofer, V. Pinti, C. R. Hoyle, and T. Peter, Atmos. Chem. Phys. {\bf 16}, 11177 (2016). 

%active sites on K-feldspar + comparisons of feldspars

\bibitem{ZBHB15} T. Zolles, J. Burkart, T. H{\"a}usler, B. Pummer, R. Hitzenberger, and H. Grothe, J. Phys. Chem. A {\bf 119}, 2692 (2015). 

\bibitem{WHKK17} T. F. Whale, M. A. Holden, A. N. Kulak, Y.-Y. Kim, F. C. Meldrum, H. K. Christenson, and B. J. Murray, Phys. Chem. Chem. Phys. {\bf 19}, 31186 (2017). 

\bibitem{HWTO19} M. A. Holden, T. F. Whale, M. D. Tarn, D. O'Sullivan, R. D. Walshaw, B. J. Murray, F. C. Meldrum, H. K. Christenson, Sci. Adv. {\bf 5} eaav4316 (2019). 

%active site
\bibitem{KBPC17} A. Kieslev, F. Bachmann, P. Pedevilla, S. J. Cox, A. Michaelides, D. Gerthsen, and T. Leisner, 
Science {\bf 355}, 367 (2017). 

%review of nuclestion 
\bibitem{SCCFPZM16} G. C. Sosso, J. Chen, S. J. Cox, M. Fitzner, P. Pedevilla, A. Zen, and A. Michaelides, 
Chem. Rev. {\bf 116}, 7078 (2016). 

%homogeneous nucleation 
\bibitem{MM10} E. B. Moore and V. Molinero, J. Chem. Phys. {\bf 132}, 244504 (2010). 

\bibitem{ESVV14} J. R. Espinosa, E. Sanz, C. Valeriani, and C. Vega, J. Chem. Phys. {\bf 141}, 18C529 (2014).

\bibitem{HD15} A. Haji-Akbari and P. G. Debenedetti, Proc. Natl. Acad. Sci. {\bf 112}, 10582 (2015). 

%heterogeneous nucleation 

\bibitem{LHV14} L. Lupi, A. Hudait, and V. Molinero, J. Am. Chem. Soc. {\bf 136}, 3156 (2014). 

\bibitem{QOHM17} Y. Qiu, N. Odendahl, A. Hudait, R. Mason, A. K. Bertram, F. Paesani, P. J. DeMott, and V. Molinero, J. Am. Chem. Soc. {\bf 139}, 3052 (2017). 

\bibitem{CKSM15} S. J. Cox, S. M. Kathmann, B. Slater, and A. Michaelides, J. Chem. Phys. {\bf 142}, 184705 (2015). 

%AgI
\bibitem{ZBP15a} S. A. Zielke, A. K. Bertram, and G. N. Patey, J. Phys. Chem. B {\bf 119}, 9049 (2015).

\bibitem{FD14} G. Fraux and J. P. K. Doye, J. Chem. Phys. {\bf 141}, 216101 (2014). 

%kaolinite

\bibitem{CRKSM14} S. J. Cox, Z. Raza, S. M. Kathmann, B. Slater, and A. Michaelides, Faraday Discuss. {\bf 167}, 389 (2014).

\bibitem{ZBP15b} S. A. Zielke, A. K. Bertram, and G. N. Patey, J. Phys. Chem. B {\bf 120}, 1726 (2015).

\bibitem{Z16} S. A. Zielke, {\it Molecular Dynamics Simulations of Heterogeneous Ice Nucleation by Atmospheric Aerosols} (Ph.D. thesis, University of British Columbia, 2016). 

%DFT
\bibitem{PCSM16} P. Pedevilla, S. J. Cox, B. Slater, and A. Michaelides, J. Phys. Chem. C {\bf 120}, 6704 (2016). 

%%%%%

%expt structure
\bibitem{KR87} H. Kroll and P. H. Ribbe, Am. Mineral., {\bf 72}, 491 (1987).

\bibitem{PK10} H. R. Pruppacher and J. D. Klett, {\it Microphysics of Clouds and Precipitation} (Springer, New York, 2010).

%CLAYFF
\bibitem{CLK04} R. T. Cygan, J.-J. Liang, and A. G. Kalinichev, J. Phys. Chem. B {\bf 108}, 1255 (2004).

%TIP4P/Ice
\bibitem{ASFV05} J. L. F. Abascal, E. Sanz, R. G. Fern\'{a}ndez, and C. Vega, J. Chem. Phys. {\bf 122}, 234511 (2005). 

\bibitem{FAV06} R. G. Fern\'{a}ndez, J. L. F. Abascal, and C. Vega, J. Chem. Phys. {\bf 124}, 144506 (2006).

%GROMACS
\bibitem{AMSS15} M. J. Abraham, T. Murtola, R. Schulz, S. P\'{a}lla, J. C. Smith, B. Hessa, and E. Lindahl,
SoftwareX {\bf 1–2}, 19 (2015).

%Nose/Hoover
\bibitem{N84} S. A. No\'{s}e, J. Chem. Phys. {\bf 81}, 511 (1984). 

\bibitem{H85} W. G. Hoover, Phys. Rev. A {\bf 31}, 1695 (1985).

%PME
\bibitem{DYP93} T. Darden, D. York, and L. Pedersen, J. Chem. Phys. {\bf 98}, 10089 (1993).

\bibitem{EPBD95} U. Essmann, L. Perera, M. L. Berkowitz, T. Darden, H. Lee, and L. G. Pedersen, J. Chem. Phys. {\bf 103}, 8577 (1995).

%LINCS
\bibitem{HBBF97} B. Hess, H. Bekker, H. J. C. Berendsen, J. G. E. M. Fraaije, J. Comput. Chem. {\bf 18}, 1463 (1997). 

%CHILL
\bibitem{MLWSM10} E. B. Moore, E. de la Llave, K. Welke, D. A. Scherlis, and V. Molinero, Phys. Chem. Chem. Phys. {\bf 12}, 4124 (2010).

%SPC water
\bibitem{BPGH81} H. J. C. Berendsen, J. P. M. Postma, W. F. van Gunsteren, and J. Hermans, in {\it Intermolecular Forces}, B. Pullman Ed. (Springer Netherlands, 1981), pp. 331-342.

%vega spc ice
\bibitem{VSA05} C. Vega, E. Sanz, and J. L. F. Abascal, J. Chem. Phys. {\bf 122}, 114507 (2005).

%expt. sim. on feldspar surfaces

\bibitem{FTGH00} P. Fenter, H. Teng, P. Gesissb\"{u}hler, J. M. Hanchar, K. L. Nagy, and N. C. Sturchio, Geochim. Cosmochim. Acta {\bf 64}, 3663 (2000). 

\bibitem{FCPZ03} P. Fenter, L. Cheng, C. Park, Z. Zhang, and N. C. Sturchio, Geochim. Cosmochim. Acta {\bf 67}, 4267 (2003). 

\bibitem{KLI08} S. Kerisit, L. Chongxuan, and E. S. Ilton, Geochim. Cosmochim. Acta {\bf 72}, 1481 (2008). 

\bibitem{QM18} Y. Qiu and V. Molinero, J. Chem. Phys. Lett. {\bf 9}, 5179 (2018). 

\end{thebibliography}

\bibliographystyle{aip}

\clearpage

%% List of Tables

%Table 1
\begin{table}[]
\centering

\begin{tabular}{ccccc}

Lattice parameter & Feldspar-a (0.8592 nm) & Feldspar-b (1.2963 nm) & Feldspar-c (0.7222 nm) \\ \hline
I$_{h}$-a (0.448 nm) & [1:2] 4.1\% & [1:3] 4.5\% & [2:3] 6.4\% \\
I$_{h}$-c (0.731 nm) & [2:3] 21.7\% & [1:2] 11.0\% & [1:1] 1.2\% \\
I$_{c}$ (0.637 nm) & [2:3] 11.0\% & [1:2] 2.0\% & [1:1] 13.4\%

\end{tabular}
\caption{Lattice mismatch percentages for feldspar with hexagonal, I$_{h}$, and cubic, I$_{c}$, ice. The lattice parameters used for feldspar and ice are given in parentheses, and the values of $n$ and $m$ in Eq. (\ref{eq:mismatch}) are given in square brackets. Note that for I$_{h}$ crystal axes $a$ and $b$ are equivalent, and for I$_{c}$ all three crystal axes are equivalent.}
\label{tab:latticematch}
\end{table}

%Table 2
\begin{table}[]
\centering

\begin{tabular}{cccccccc}

Surface Label & X(nm)  & Y(nm)  & Z(nm)  & N           & $\rho$ (g/cc)       &   Trials         & Time (ns) \\ \hline
(001)   & 5.1379 & 5.1809 & 8.2947 & 4070            & 0.945         & 2           & 1200              \\
(010)   & 5.0578 & 4.6124 & 9.5846 & 3640            & 0.940         & 2           & 850               \\ 
(100)a  & 5.2435 & 4.3903 & 8.7700 & 3810            & 0.955         & 3           & 975               \\
(100)b  & 5.2435 & 4.3903 & 8.9760 & 3600            & 0.950         & 2           & 700               \\
(100)c  & 5.2435 & 4.3903 & 8.7700 & 3600            & 0.945         & 4           & 900               \\
Defect  & 5.9942 & 6.4654 & 9.4122 & 5680            & 0.950         & 2           & 900              

\end{tabular}
\caption{Summary of the simulations performed together with some simulation parameters. X, Y, and Z are the dimensions of the simulation cell, N is the number of water molecules in the central cell, $\rho$ is the bulk liquid density at 300 K, Trials are the number of simulations run with different initial conditions, and Time is the length of the simulation carried out at 230 K. Note that Z includes two vacuum spaces, each 1.0 nm thick, as measured from the boundary of the slab unit cell.}
\label{tab:simsummary}
\end{table}

%fig 1
\clearpage
\begin{figure}[!htb]
    \centering
   %\includegraphics[width\=linewidth]{topviewslabcorrect.eps}
   \begin{tabular}{@{}cc@{}}
   (001) & (010) \\
   	\includegraphics[trim={2cm 2cm 2cm 4cm},clip,width=.48\textwidth]{001topviewveryfinal.eps} &
   	\includegraphics[width=.50\textwidth]{010topviewslab.eps} \\
   (100)a & (100)b \\	
   	\includegraphics[trim={0 0 0 2cm},clip,width=.50\textwidth]{100atopviewfinal.eps} &
   	\includegraphics[trim={0 0 0 2cm},clip,width=.50\textwidth]{100btopviewslabfinal.eps} \\
    \end{tabular}
   \caption{Top views of the (001), (010), (100)a, and (100)b K-feldspar (microcline) surfaces. The appearance of (100)c (not shown) is very similar to (100)a. Silicon, aluminium, potassium, and oxygen atoms are colored yellow, cyan, pink, and red, respectively. Hydroxyl hydrogen atoms are black.} 
      \label{fig:topview}
\end{figure}

%fig 2
\clearpage
\begin{figure}
    \centering 
     \includegraphics[width=.80\textwidth]{chillfinal.eps}
   \caption{Total number of ``ice molecules'' detected by the CHILL algorithm at 230 K as a function of time. The color codes for the different surfaces are indicated in the legend. Note that ice nucleation is not indicated for any surface at 230 K.}
      \label{fig:chill}
\end{figure}

%fig 3
\clearpage
\begin{figure}[!htb]
    \centering
    \begin{tabular}{@{}cc@{}}
    (001) & (010)\\
   	\includegraphics[trim={1cm 1cm 1cm 1cm},clip,width=.40\textwidth]{001slabwaterfinal.eps} &
   	\includegraphics[trim={1cm 1cm 1cm 1cm},clip,width=.40\textwidth]{010waterslabfinal.eps} \\
   	(100)a & (100)b \\
   	\includegraphics[trim={1cm 1cm 1cm 1cm},clip,width=.40\textwidth]{100aslabwaterfinal.eps} &
   	\includegraphics[trim={1cm 1cm 1cm 1cm},clip,width=.40\textwidth]{100bwaterslabfinal.eps} \\
   	\end{tabular}
    \caption{Snapshots of the final configurations for the (001), (010), (100)a, and (100)b surfaces at 230 K. The oxygen and hydrogen atoms of water are red and black, respectively. The colors of the atoms in the feldspar slabs are as in Fig. \ref{fig:topview}.}
     \label{fig:finalsnapshots}
\end{figure}

%fig 4
\clearpage
\begin{figure}
    \centering
    \begin{tabular}{@{}cc@{}}
    (001) & (010) \\[4pt]
    	\includegraphics[width=.50\textwidth]{001density.eps} &
   	\includegraphics[width=.50\textwidth]{010density.eps} \\
   	(100)a & (100)b \\[4pt]
   	\includegraphics[width=.50\textwidth]{100adensity.eps} &
   	\includegraphics[width=.50\textwidth]{100bdensity.eps} \\
   	\end{tabular}
    \caption{Density profiles for the (100), (010), (100)a, and (100)b surfaces at 230 K. Profiles for water oxygen and water hydrogen atoms are red and black, respectively. Profiles for the surface hydroxyl hydrogen atoms are green.}
     \label{fig:densities}
\end{figure}

%fig 5
\clearpage
\begin{figure}
    \centering
    \begin{tabular}{@{}cc@{}}
    (001) & (010) \\
   	\includegraphics[trim={1cm 1cm 1cm 1cm},clip,width=.48\textwidth]{001water.eps} &
   	\includegraphics[trim={2cm 2cm 2cm 3cm},clip,width=.48\textwidth]{010water.eps} \\
   	(100)a & (100)b  \\
   	\includegraphics[trim={0.5cm 0.5cm 0.5cm 2.5cm},clip,width=.48\textwidth]{100awater.eps} &
   	\includegraphics[trim={0.0cm 0.0cm 1.5cm 3cm},clip,width=.48\textwidth]{100bwater.eps} \\
   	\end{tabular}
    \caption{The structure of the surface water layer for (001), (010), (100)a, and (100)b at 230 K. Water oxygen and hydrogen atoms are red and black, respectively. For clarity the atoms of the feldspar surfaces are not shown.}
     \label{fig:surfacestructure}
\end{figure}

%fig 6
\clearpage
\begin{figure}
    \centering
    \includegraphics[trim={0 0.7cm 0 3cm},clip,width=.90\textwidth]{Figure7.eps}\vspace{-0.4cm}
    \newline
    \quad\quad 50 K       \quad\quad\quad\quad\quad\quad\quad\quad\quad\quad\quad 100 K \quad\quad\quad\quad\quad\quad\quad\quad\quad\quad\quad\quad 150 K \\
    \caption{Final snapshots of the two ice layers after MD simulations at 50, 100, and 150 K. At 50 K disordering was not observed during a simulation of 150 ns. The atoms are colored as in Figs. \ref{fig:topview} and \ref{fig:finalsnapshots}.}
      \label{fig:2icelayers}
\end{figure}

%fig 7
\clearpage
\begin{figure}
    \centering
   \includegraphics[trim={0 0.8cm 0 3cm},clip,width=.90\textwidth]{Figure9.eps}\vspace{-0.8cm}
   \newline
    0 ns  \quad\quad\quad\quad\quad\quad\quad\quad 2 ns \quad\quad\quad\quad\quad\quad\quad\quad 5 ns \quad\quad\quad\quad\quad\quad\quad 10 ns \\
    \caption{Snapshots showing four layers of ice becoming disordered at 230 K. The atoms are colored as in Figs. \ref{fig:topview} and \ref{fig:finalsnapshots}.}
    \label{fig:4icelayers}
\end{figure}

%fig 8
\clearpage
\begin{figure}
    \centering
    \includegraphics[trim={0 0.8cm 0 0},clip,width=.90\textwidth]{Figure10.eps}\vspace{-0.7cm}
    \newline
    \quad\quad 0 ns       \quad\quad\quad\quad\quad\quad\quad\quad\quad\quad\quad 150 ns \quad\quad\quad\quad\quad\quad\quad\quad\quad 325 ns \\
    \caption{Snapshots from a MD simulation initiated with two rigid ice layers in contact with liquid water at 230 K. The atoms are colored as in Figs. \ref{fig:topview} and \ref{fig:finalsnapshots}.}
    \label{fig:2rigidicelayers}
\end{figure}

%fig 9
\clearpage
\begin{figure}
    \centering
    \includegraphics[trim={0 12cm 0 0},clip,width=.90\textwidth]{Figure11.eps}\vspace{-0.65cm}
    \newline
    \quad\quad 0 ns       \quad\quad\quad\quad\quad\quad\quad\quad\quad\quad 40 ns \quad\quad\quad\quad\quad\quad\quad\quad\quad\quad 400 ns \\
    \caption{Snapshots from a simulation initiated with two flexible ice layers in contact with water at 230 K. The atoms are colored as in Figs. \ref{fig:topview} and \ref{fig:finalsnapshots}.}
    \label{fig:2flexibleicelayers}
\end{figure}

%fig 10
\clearpage
\begin{figure}
    \centering
    \includegraphics[trim={0 0.8cm 0 0},clip,width=.90\textwidth]{Figure13.eps}\vspace{-0.89cm}
    \newline
    \quad\quad 0 ns       \quad\quad\quad\quad\quad\quad\quad\quad\quad\quad 300 ns \quad\quad\quad\quad\quad\quad\quad\quad\quad\quad 670 ns \\
    \caption{Snapshots from a simulation initiated with four flexible ice layers in contact with water at 230 K. The atoms are colored as in Figs. \ref{fig:topview} and \ref{fig:finalsnapshots}.}
    \label{fig:4flexibleicelayers}
\end{figure}

%fig 11
\clearpage
\begin{figure}
    \centering
    \includegraphics[trim={0 0.8cm 0 0},clip,width=.90\textwidth]{Figure14f.eps}\vspace{-0.75cm}
    \newline
   Initial Snapshot          \quad\quad\quad\quad\quad\quad\quad\quad\quad\quad\quad\quad\quad\quad Final Snapshot
    \caption{Initial and final snapshots of ice between two (100)c slabs. During the 24 ns simulation at 230 K, the two originally rigid ice layers in contact with the bottom slab are allowed to move, and quickly disorder as shown in the final snapshot. The atoms are colored as in Figs. \ref{fig:topview} and \ref{fig:finalsnapshots}.}
        \label{fig:snapshotsinitialfinal}
\end{figure}

%fig 12
\clearpage
\begin{figure}
    \centering
   \includegraphics[width=.80\textwidth]{compareall.eps}
    \caption{Initial and final water oxygen density profiles of ice between two (100)c slabs. The initial profile is obtained holding two rigid ice layers at one surface, and the final profile is obtained after a 24 ns simulation at 230 K where all water molecules are allowed to move. The profiles correspond to the initial and final snapshots shown in Fig. \ref{fig:snapshotsinitialfinal}. The oxygen density profile for liquid water at 230 K in contact with the same slabs is included for comparison. We see that the profile of the disordered ``ice'' near the surface resembles the profile of liquid water.}    
     \label{fig:oxygendensity}
\end{figure}

%fig 13
\clearpage
\begin{figure}
    \centering
\includegraphics[trim={1cm 1cm 1cm 1cm},clip,width=.40\textwidth]{defectsnapshot.eps}    
          \caption{Snapshot of the final configuration of a model ``defect'' simulation. The (001) surface lies in the XY plane, and the (100)a surface at an angle consistent with the triclinic unit cell of microcline. The atoms are colored as in Figs. \ref{fig:topview} and \ref{fig:finalsnapshots}.}
         \label{fig:defectsnapshot}
\end{figure}

\end{document}
\end{document}
