\documentclass[notitlepage,letterpaper,onecolumn,12pt,final]{article}
\pdfpageheight=11in
\pdfpagewidth=8.5in

\usepackage[margin=0.75in,tmargin=0.75in,headheight=18pt,headsep=\baselineskip]{geometry}
%Customizing the header
\usepackage{fancyhdr}
\pagestyle{fancyplain}
\fancyhf{}
\renewcommand{\headrulewidth}{0pt} % remove lines as well
\rhead{Hayden Scheiber - PIN 540569}
%End header

\usepackage[T1]{fontenc}
\usepackage[utf8]{inputenc}
\usepackage{paralist}
\usepackage{xpatch}
\usepackage[style=numeric,maxcitenames=1,uniquelist=false,backend=biber,sorting=none]{biblatex}
\addbibresource{Research_Proposal.bib}

\defbibenvironment{bibliography}
  {\inparaenum[ \printfield{labelnumber}.]}
  {\endinparaenum}
  {\item}
\DeclareFieldFormat[article,periodical]{volume}{\textbf{#1}}
\DeclareFieldFormat{pages}{#1}
\DeclareNameAlias{default}{last-first}
\renewbibmacro{in:}{}
\xpatchbibmacro{journal+issuetitle}
  {\usebibmacro{volume+number+eid}}
  {\usebibmacro{volume+number+eid}%
   \setunit{\addcomma\space}%
   \printfield{pages}%
   \clearfield{pages}}
  {}{}

\defbibenvironment{bibliography}
  {\inparaenum[ \textbf{[\printfield{labelnumber}]}]}
  {\endinparaenum}
  {\clearfield{title}\item}

% Square brackets in supercite
\DeclareCiteCommand{\supercite}[\mkbibsuperscript]
  {\iffieldundef{prenote}
     {}
     {\BibliographyWarning{Ignoring prenote argument}}%
   \iffieldundef{postnote}
     {}
     {\BibliographyWarning{Ignoring postnote argument}}%
   \bibopenbracket}%
  {\usebibmacro{citeindex}%
   \usebibmacro{cite}}
  {\supercitedelim}
  {\bibclosebracket}

\linespread{0.85}

% Times Roman font
\usepackage{mathptmx}
\bibliography{Research_Proposal.bib}

% Remove references title
\renewcommand\refname{}

% No indents
%\setlength\parindent{3pt}

\usepackage{lipsum}

\begin{document}
%
\noindent\textbf{Proposed Research: Exploration of Crystal Nucleation Phenomena Through Molecular Simulation}
\smallskip

There exists a vast set of chemically interesting questions that are extremely difficult to explore experimentally with existing methods. The difficulty in such experiments frequently arises due to the tiny length scales and/or very rapid time scales of the phenomena in question. On the other hand, computer simulation of chemical systems is a field that has flourished in recent years due to a combination of compounding factors. The rigorous theoretical framework continually developed over the last hundred years, combined with the recent rapid development and advancement in computational power, has allowed computer simulation of dynamical chemical systems---known as molecular dynamics (MD)---to reach a new level of scientific possibility. We are now at a point where the current generation of theoretical chemists can confidently investigate chemically complex systems with high accuracy, often in unexplored domains where physical experiments are most difficult to perform.

The current proposal is concerned with a particularly formidable research topic: the nucleation mechanism of salts in water. Nucleation is defined as the first step in the spontaneous formation of new thermodynamic phases; it is a particularly troublesome process to explore experimentally due to the typical tiny size of nuclei combined with the complexity of the process. An example of the difficulty in exploring this process can be found in the direct measurement of nucleation growth rates experimentally. Proper measurement is challenging because crystal growth rates typically vary substantially with the size of the nucleus, and only larger crystals can be easily observed. Nucleation phenomena are conventionally described by classical nucleation theory (CNT).~\supercite{Karthika2016} CNT posits that the formation of nuclei is governed by two interactions: surface tension at the border of the nucleus, and internal interactions which are assumed equivalent to the bulk crystal. CNT is usually inadequate to fully describe the details of nucleation.~\supercite{Karthika2016} For instance, CNT predicts that nucleus formation is a one-step process but, in the case of lithium halides, preliminary MD studies show this not to be the case.~\supercite{Lanaro2018} A complete mechanistic description of nucleation would have profound implications on a wide range of fields. Such fields include: chemical engineering, where nucleation is often undesirable; biochemistry, where protein crystallization is extremely important for X-ray characterization; and climate science, where water nucleation plays a critical role. In my research, I will explore this fundamental topic through MD investigation of lithium halides, as these are the simplest systems that likely nucleate through a non-classical (inconsistent with CNT) pathway.~\supercite{Lanaro2018}

In MD simulations, empirical force fields (or \emph{models}) are carefully chosen to govern the interactions between individual atoms. Presently, the models available for implementing MD simulations of lithium halides are seriously flawed; when executed in MD simulations, the models predict an incorrect crystal structure.~\supercite{Lanaro2017a} These models were originally adjusted to correctly predict bond lengths and crystal lattice energies for the experimentally verified \emph{rocksalt} crystal structure. However, the creators assumed---without testing---that the rocksalt structure would be the lowest energy (i.e. most stable) state. Instead, it was recently found~\supercite{Lanaro2017a} that the \emph{wurtzite} structure is more stable for the current models. Without prediction of the proper crystal structure, these models cannot be utilized to accurately explore more chemically complex nucleation phenomena; hence the first step of my research is to improve the current empirical models.

How can theoretical models be improved when interactions between single atoms cannot be measured? My approach is to use quantum mechanics (QM), the robust theoretical framework that forms the basis for much of modern chemistry. This research is well underway, and I have already found that QM calculations accurately predict bond lengths, crystal energies, and crystal structures for lithium halides. The next step is to reparametrize the existing empirical models to fit the QM data. If this is not possible, I will create new empirical models with different characteristics (such as explicit many-body interactions) to match the QM results. Such a modification would increase the computational complexity of the MD simulations, but may be necessary to correctly model non-classical nucleation mechanisms. After improving the models, the next step is to elucidate the mechanism of lithium halide nucleation in water through MD simulation. The MD data will be investigated by a multitude of statistical analysis techniques. Related to this area of research, there is a peculiar property of lithium fluoride that has been confirmed experimentally:~\supercite{Booth1950} while most salts increase in solubility at higher temperature, lithium fluoride solubility in water decreases. The mechanism for this phenomena is currently unknown, but is a well-suited question for me to consider. I plan to use my improved empirical models to explore this anomaly with unique precision. 

\printbibliography[heading=none]
\end{document}