\documentclass[titlepage]{article}

\usepackage{amsmath, amssymb}
\usepackage{float}
\usepackage{gensymb}
\usepackage[pdftex]{graphicx}
\usepackage[margin=1in]{geometry}
\usepackage{ragged2e}
\usepackage{setspace}
\usepackage{subfigure}
\usepackage{wrapfig}




\begin{document}
%----------------------------------------------------------------------------------------
%	TITLE PAGE
%----------------------------------------------------------------------------------------

\begin{titlepage} % Suppresses displaying the page number on the title page and the subsequent page counts as page 1
	\newcommand{\HRule}{\rule{\linewidth}{0.5mm}} % Defines a new command for horizontal lines, change thickness here
	
	\center % Centre everything on the page
	\begin{figure*}
		\centering
		\includegraphics[width=\textwidth]{figures/UBC_Chemlogo.eps}
	\end{figure*}
	
	
	%------------------------------------------------
	%	Headings
	%------------------------------------------------
	
	\textsc{\LARGE Fourth Year Meeting\\ \vspace{1em}  \& Thesis Outline}\\[1.5cm] % Main heading such as the name of your university/college
	
	
	%------------------------------------------------
	%	Title
	%------------------------------------------------
	
	\HRule\\[0.4cm]
	
	{\huge\bfseries Exploration of Crystal Nucleation Phenomena Through Molecular Simulation}\\[0.4cm] % Title of your document
	
	\HRule\\[1.5cm]
	
	%------------------------------------------------
	%	Author(s)
	%------------------------------------------------
	
	\begin{minipage}{0.4\textwidth}
		\begin{flushleft}
			\large
			\textit{Author}\\
			Hayden \textsc{Scheiber} % Your name
		\end{flushleft}
	\end{minipage}
	~
	\begin{minipage}{0.4\textwidth}
		\begin{flushright}
			\large
			\textit{Supervisor}\\
			Dr. Gren \textsc{Patey} % Supervisor's name
		\end{flushright}
	\end{minipage}\\

	\vspace{1cm}
	
	%\begin{minipage}{0.4\textwidth}
	%	\begin{flushleft}
	%		\large
	%		\textit{Chair}\\
	%		Dr. Geoffrey \textsc{Herring} % Chair
	%	\end{flushleft}
	%\end{minipage}
	~
	\begin{minipage}{0.4\textwidth}
		\begin{center}
			\large
			\textit{Committee}\\
			Dr. Yan \textsc{Wang}\\ % Committee
			Dr. Mark \textsc{Thachuk}\\ % Committee
			Dr. Keng \textsc{Chou} % Committee
		\end{center}
	\end{minipage}
	
	%------------------------------------------------
	%	Date
	%------------------------------------------------
	
	\vfill\vfill\vfill % Position the date 3/4 down the remaining page
	
	{\large
	\begin{table}[H]
		\centering
		\begin{tabular}{lll}
			\textbf{Date} & \textbf{Time} & \textbf{Location} \\
			Monday, October 4$^{\text{th}}$ 2021 & 11:00 AM & CHEM D317
		\end{tabular}
	\end{table}
	}
	%------------------------------------------------
	%	Logo
	%------------------------------------------------
	
	%\vfill\vfill
	%\includegraphics[width=0.2\textwidth]{placeholder.jpg}\\[1cm] % Include a department/university logo - this will require the graphicx package
	 
	%----------------------------------------------------------------------------------------
	
	\vfill % Push the date up 1/4 of the remaining page
	
\end{titlepage}

%----------------------------------------------------------------------------------------

\large
\justifying
\doublespacing

\section{Introduction}
Aerosols play an important role in the atmosphere affecting earth's climate, weather and human health.  Aerosols can come from a variety of natural and anthropogenic sources and range in size from single molecules to dust particles and bacteria\cite{SP06}.  Interaction with water groups aerosols  into two main classes, cloud condensation nuclei that allow water to condense into clouds and ice nuclei(IN) that freeze water at temperatures above -38\celsius.  \\
\indent Although  a common process, how water freezes is still a problem\cite{BR13}.  Water melts at 0\celsius, but pure water droplets will not freeze homogeneously until -38\celsius\cite{BR12}.  Water can freeze heterogeneously at higher temperatures with the aid of a heterogeneous IN\cite{SP06} that acts as a catalyst and lowers the Gibbs free energy of nucleation.  IN form a minor component of atmospheric aerosols with estimates that approximately only 1/1000000 aerosol particles are active ice nuclei at -20\celsius.  Yet ice nuclei play an important role in weather, for instance most rain begins as ice crystals, which grow large enough to fall via the Bergeron-Findeisen process, where ice crystals grow at the expense of water droplets.  Given the selective nature of effective ice nuclei various requirements have been proposed to explain the effectiveness of some particles over others.  IN need to be insoluble in water to provide a ridged structure on which ice can nucleate\cite{PK10}.  Particles need to  be larger than the critical ice embryo to initiate nucleation.  The IN must be able to bond to water, specifically in the form of hydrogen bonds.  Crystal structures of ice and the IN need to match.  This mismatch between between structures is quantified in equation~\ref{matcheq},

\begin{equation}
\delta=\frac{na_{o,n}-ma_{o,i}}{ma_{o,i}}\times 100\%
\label{matcheq}
\end{equation}

where the mismatch, $\delta$, is the percentage differing in the lattice parameters of ice, $a_{o,i}$, and the IN, $a_{o,n}$.  $m$ and $n$ are integers which are chosen to minimize $\delta$.  Good ice nuclei will generally have mismatches of less then a few percent.  Lastly there is an active site requirement of IN which may be in the form of surface defects or parts of the crystal that have a small lattice mismatch.  Freezing of water by an IN can occur via one of several mechanisms: Deposition freezing occurs with water vapour freezing directly onto an IN without significant liquid water forming, contact freezing signifies a liquid water drop encountering an IN and nucleating at their interface, and immersion freezing where an IN is submerged inside a liquid drop and nucleates freezing.\\
\indent Classical nucleation theory(CNT) provides a general theoretical framework for nucleation of one phase of a substance into another\cite{SP06, HB09}.  In CNT a nucleus is thought of as a sphere of one phase in a bulk solution of the other.  Monomers of the substance can then join or leave the nucleus one at a time.  Thermodynamically, the Gibbs free energy of nucleation is thought of as the decrease in energy as the nucleus forms competing against the increase of energy by the formation of an interface between the phases,
\begin{equation}
\Delta G(a)=n_{2}\frac{4\pi a^{3}}{3}(\Delta \mu_{1,2})+4\pi a^{2}\gamma_{1,2}
\label{CNTfree}
\end{equation}
where in the first term $\Delta \mu_{1,2}$ is the difference in chemical potential of the bulk(1) and nucleated(2) phases, which is negative, multiplied by the volume of a sphere of radius $a$ and the density of the nucleated phase $n_{2}$.  Added to the first term is the destabilization by the interface given by the area of the sphere multiplied by the interfacial energy, $\gamma_{1,2}$.  This equation gives rise a critical radius, $a*$, and an activation barrier, $\Delta G(a)*$.  A nucleus larger than the critical radius will grow larger, but a smaller one will not.\\
\indent There are about sixteen known polymorphs of ice\cite{M09}, most of which are named with a Roman numeral indicating the order in which the polymorph was found.  The most common form of ice that exists under atmospheric conditions is ice I which is usually refereed to as hexagonal ice, or I$_{h}$.  A metastable form of ice that can also exist under atmospheric conditions, and which does not have a number, is cubic ice, or I$_{c}$.  Ices II-XV are usually higher density forms of ice or proton ordered forms of other ices.  Oxygen atoms in ice are well ordered and structures of ice are generally based on the arrangement of the oxygen atoms.  Hydrogen hydrogen bonds to neighbouring water molecules, however the orientation of the water molecules, and the hydrogen order, is otherwise unordered, and even at 0K there is a residual entropy.\\
\indent I$_{h}$ and I$_{c}$ are both made of of a bilayer of water molecules and differ in their stacking pattern.  This bilayer is made of water molecules forming hexagonal rings in the chair conformation, creating two sublayers, or one bilayer.  Differences in the stacking of this bilayer results in each forms of ice.  An ABAB... stacking pattern gives I$_{h}$, and an ABCABC... stacking gives I$_{c}$.  The difference between each form can also be seen in the types of rings that are formed between each bilayer.  In I$_{h}$ the rings are boats, while in I$_{c}$ the rings are also chairs\cite{BADH08}.  Previous simulations have shown that ice prefers to grow on the bilayer\cite{C07}, an the form of ice that is growing can change from layer to layer as there is a very low interfacial energy between them\cite{J98}.\\

\section{Methods}
\subsection{Molecular dynamics}
\indent To investigate the nucleation of ice on atmospheric aerosols we employ molecular dynamics simulations.  Molecular dynamics is a classical treatment of molecules, which are represented as point particles and whose motioned is governed by Newton's equations of motion.  This treatment allows for simulations of systems that are much larger than what quantum mechanics based calculations can achieve.  A molecular dynamics simulation involves a series of steps\cite{FS02}: \\
\begin{enumerate}
\item The simulation is initialized, initial positions of particles are created and velocities are given.  Often many of the parameters for this are stored in several files which are read into the program, giving it the instructions needed to run the simulation.\\
\item Forces between each particle are calculated.  Forces are determined by equation~\ref{force}
\begin{equation}
F(r)=\sum_{i=1}^{n-1}\sum_{j=2}^{n}F(r_{ij})
\mbox{   where   }
F(r_{ij})=-\frac{dV(r_{ij})}{dr}
\label{force}
\end{equation}
The negative of the derivative of the potential energy, $V(r_{ij})$ is summed over all pairs of particles in the simulation separated by distance $r_{ij}$.  In the simulations that are described here, the potential energy in calculated according to equation~\ref{potential}.
\begin{equation}
V_{ij}(r)=\frac{q_{i}q_{j}}{4\pi\epsilon_{0}r_{ij}} + 4\epsilon\left[ \left(\frac{\sigma}{r_{ij}}\right)^{12} - \left(\frac{\sigma}{r_{ij}}\right)^6\right]
\label{potential}
\end{equation}
The first term is the coulombic interaction between particles, followed by the Lennard-Jones potential which accounts for Van der Waals forces and short range repulsion.  $\sigma$ gives the x intercept of the function and sets the size of the particle and $\epsilon$ sets the depth of the energy well causing short range attraction between particles.  Due to the summation in equation~\ref{force} this step is the most time consuming part of any simulation.\\
\item Particles are then moved by integrating Newton's equations of motion.  Several methods for doing this exist.  The traditional one is the Verlet algorithm.  Here the position of the particle in the subsequent time step is approximated as a Taylor expansion around its current position.  Added to this expansion is another Taylor expansion of the particle's position in he previous time step resulting in equation~\ref{verlet},
\begin{equation}
x(t+\Delta t) = 2x(t)-x(t-\Delta t)+\frac{F(t)}{m}\Delta t^2
\label{verlet}
\end{equation}
where $\Delta t$ is the time step and m is the mass of the particle.  This equation does a reasonable job of conserving energy in a simulation, which is required as molecular dynamics naturally simulates systems of constant energy, and is reversible in time.  The Verlet algorithm does not directly incorporate the velocity of the particles into its equation.  It is possible to derive mathematically equivalent equations that do incorporate the velocity, such as the velocity-verlet algorithm and the Leap Frog algorithm, the latter of which is used in the simulations reported here.  The Leap Frog algorithm, equations~\ref{leapvel} and \ref{leappos}, updates the velocity of the particle at half integer time steps ahead of the current time step, followed by the position 'leaping' over the velocity to the next time step. 
\begin{equation}
v(t+\Delta t/2)=v(t-\Delta t/2)+\Delta t\frac{F(t)}{m}
\label{leapvel}
\end{equation}
\begin{equation}
x(t+\Delta t)=x(t)+\Delta tv(t+\Delta t/2)
\label{leappos}
\end{equation}
\item Steps 2 and 3 are repeated until the simulation is complete at which point data calculated during the simulation is reported or additional software is used to conduct analysis of the simulation from printed trajectories or other data.  One of the primary analysis tools we use is the CHILL algorithm\cite{MLWSM10} which is used to detect which water molecules in the simulation have frozen, and into which phase of ice they belong to.  The algorithm is based on analysis of the oxygen atoms of the four nearest neighbouring water molecules to a particular water.  It is then determined if the bond to a particular neighbour is staggered or eclipsed.  If all four bonds are staggered the water is part of I$_{c}$; if only three of the bonds are staggered and one is eclipsed the water molecule is part of I$_{h}$.\\
\end{enumerate}
\indent Simulations have a finite size which can cause unrealistic results to occur.  Boundaries are 'removed'  from the simulation to give more realistic results and is accomplished with periodic boundary conditions(PBC).  With PBC, when a particle encounters a wall of the simulation cell it disappears at that wall and reappears at the opposite wall.  Force calculations are done to include periodicity as particles only interact with the nearest neighbours of other particles.\\
\indent Molecular dynamics naturally simulates a system in the micro-canonical or NVE ensemble.  However, other ensembles are often desired , so thermostats and barostats are used to produce a simulation in the correct ensemble.  In this report, simulations are done under canonical or NVT conditions and use a thermostat to maintain a constant average temperature.  There are several different thermostats that have been created, one of the simplest being the Anderson thermostat.  This thermostat couples the simulation to a heat bath via random particles that are selected and have their velocities are changed to lie within a Maxwell-Boltzmann distribution around the desired temperature.  This method produces the correct ensemble for equilibrium properties of the system, however the random changing of a particles velocity ruins any dynamical data, such as diffusion, that can be obtained.  There are however other thermostats that produce the correct ensemble and keep the dynamics of the simulation deterministic.\\
\indent One such thermostat that can produce the correct ensemble without disrupting the dynamics of the simulation is the Nose-Hoover thermostat\cite{FS02, AT87}.  Here additional terms are added to the Lagrangian of the simulation to couple the simulation to a heat reservoir,
\begin{equation}
\mathcal{L}=\sum_{i=1}^{N}\frac{m_{i}}{2}s^{2}\dot{r}_{i}^{2}-V(r^{N})+\frac{Q}{2}\dot{s}^{2}-({\it f}+1)k_{B}Tlns
\label{NHlagrang}
\end{equation}
where the first and second term are the kinetic and potential energies of the real system and the third and fourth are regarded as the kinetic and potential energies of the reservoir.  $s$ is and additional coordinate from the heat reservoir, $Q$ determines the strength coupling of the system to the reservoir and ${\it f}$ is the number of degrees of freedom in the system.  From this the equations of motion can be derived
\begin{equation}
\ddot{r}=\frac{a}{s^{2}}-\frac{2\dot{s}\dot{r}}{s}
\end{equation}
\begin{equation}
Q\ddot{s}=\sum_{i}m\dot{r}_{i}^{2}s-\frac{({\it f}+1)k_{B}T}{s}
\end{equation}

%Simulation details**********
\subsection{Simulation details}
	To carry out the simulations we employed the GROMACS 4.5.5 molecular dynamics software package\cite{P13} as it can easily handle large simulations.  Water was represented with two different models to ensure that the results are independent of the model for water chosen.  These models were TIP4P/Ice\cite{ASFV05} and the six-site model\cite{NE03} which have I$_{h}$ melting temperatures of ~270K and ~289K\cite{AFVC06, ASFV05} respectively.  Each model represents oxygen as a Lennard-Jones particle, hydrogen as a positive point charge and have a negative point charge located near the oxygen bisecting the HOH bond angle.  In addition to this, the six-site model also includes Lennard-Jones potentials on hydrogen and `lone pairs' in the form of negative point charges.  Both models were developed with the intention of simulating low temperature water and ices.   Analysis of the molecular dynamics trajectories was done with software provided with GROMACS for density calculations and home-made software for other analysis such as the CHILL algorithm. \\
\indent In the simulation cell two slabs of the crystal being studied are placed facing each other to cancel electric fields produced by them.  The slabs meet the boundaries in such a way that the periodicity continues the crystal, and as such produces an infinite surface.  Water molecules are placed between the slabs to achieve a density between 0.92 and 0.96g/mL.  On the other side of the slabs a gap of 0.5nm is left between the slab and the bottom and top of the cell.  This allows the use of 3D particle mesh ewald to account for long range electrostatics in the simulation and preserve a simulation that is only infinite in two dimensions.  Generally the positions of the ions in the crystal are fixed and only the water is allowed to move.
%Results section************
\section{Results and Discussion}
\subsection{Infinite AgI Slabs}


\indent Silver iodide is one of the best ice nuclei know\cite{PK10}, nucleating ice at temperatures as warm as -3\celsius, and has been used as a cloud seeding agent\cite{B99}.  The experimental discovery of silver iodide as an IN was due to its small lattice mismatch with ice\cite{V47}.  We started research with silver iodide as it is such an effective IN and likely represents an ideal case to gain insight into the nucleation of ice by aerosols, and will shed light on the mechanism used by other materials.\\  
\indent Under atmospherically relevant conditions there are two phases of silver iodide that can exist\cite{HK99}.  $\beta$-AgI is the most stable phase and adopts a hexagonal wurtzite structure.  A metastable phase, $\gamma$-AgI, also exists under atmospherically relevant conditions with a face-centred cubic structure and half of the tetrahedral holes filled by the counter ion.  These structures are related by the $\beta$-AgI(0001) and $\gamma$-AgI(111) faces which are almost identical.  Each face is made of chair conformation hexagonal rings of alternating silver and iodide ions forming a bilayer, with either all the silver or all the iodide ions occupying the upper part of the chair and the other ion the below.  Stacking of this bilayer in an ABAB... fashion gives $\beta$-AgI and an ABCABC... stacking gives $\gamma$-AgI.  Cleavage along this plane will gives either a silver or iodide exposed face depending on which ion forms the outermost part of the bilayer.\\
\begin{wrapfigure}{lhtp}{8cm}
	\begin{center}
		\includegraphics[width=8cm]{growth.pdf}
		\singlespacing\caption{Several snapshots of ice growing on AgI.  The left column shows growth on the $\beta$-AgI(0001) face at 0, 11 and 36ns from the time the simulation reached 270K, from top to bottom.  The right column shows growth on the $\gamma$-AgI(001) face at 0, 3 and 6ns from the time the simulation reached 270K.  Silver and iodide ions, hydrogen and oxygen of liquid water are silver, green, black and red respectively.  Oxygen of frozen water is blue for I$_{h}$ and yellow for I$_{c}$.}
	\label{growthpic}
	\end{center}
	
\end{wrapfigure}
\indent Given how similar the polymorphism of silver iodide and ice are, several faces of silver iodide have lattice mismatches with both I$_{h}$ and I$_{c}$ under a few percent.  Of these faces the most studied theoretically is the $\beta$-AgI(0001) face\cite{TH93, S05}.  Several previous simulations have also included some of the prism faces of $\beta$-AgI, however we do not know of any simulations of water on $\gamma$-AgI.  In our simulations we simulated the silver and iodide exposed $\beta$-AgI(0001), $\gamma$-AgI(111) and $\gamma$-AgI(001) faces and the $\beta$-AgI(10$\bar{1}$0) face.  Ice nucleation was only observed on the silver exposed $\beta$-AgI(0001), $\gamma$-AgI(111) and $\gamma$-AgI(001) faces.  The results presented below are all from simulations with the six-site model of water, however similar results were observed for TIP4P/Ice water.  In general, six-site water nucleated and grew ice faster than TIP4P/Ice, and the warmest observed nucleation of ice on the silver exposed $\beta$-AgI(0001) face was at 281K for six-site and 271K for TIP4P/Ice.  Nucleation took longer times at higher temperatures, and it is predicted that the highest nucleation temperature may be a few degrees higher if the simulations were run long enough.\\
\indent Fig.~\ref{growthpic} displays snapshots of ice growing on the silver exposed $\beta$-AgI(0001) and $\gamma$-AgI(001) faces.  The left column, showing growth on the $\beta$-AgI(0001), face shows growth of bilayers of ice parallel to the crystal surface.  Both phases of ice are present and meet up well\cite{J98}.  Growth in this direction, on the I$_{h}$(0001) and I$_{c}$(111) faces, has been observed to be the preferred growth direction of ice\cite{C07}.  In the simulation, crystals growing on each face meet at around 11ns with a layer of high density water between them.  After the meeting, the lower crystal then began to melt as the upper one grew until the liquid layer is on the crystal face.\\
\indent Growth on the the silver exposed $\gamma$-AgI(001) face is shown in the right column of Fig.~\ref{growthpic}.  Almost all of the ice that was formed was cubic, and the layers of ice are oriented at an angle to the surface, not parallel.  Given the cubic motif of this face, it is reasoned that only cubic ice can meet the surface in an energetically favourable fashion.  The lack of I$_{h}$ may be due to the size of this system, that it is too small for a layer of I$_{h}$ to be stable as it would have to terminate on the silver iodide surface.  I$_{h}$ may be able to form in larger systems, as has been seen in simulations of ice nucleation with electric fields\cite{YP13}.

\begin{figure}[htp]
	\begin{center}
		\subfigure[The first layer of water near the silver and iodide exposed faces, left and right respectively.]{\label{betaface}\includegraphics[height = 6cm]{betaface.pdf}}
		\subfigure[Normalized density profile of oxygen, red, and hydrogen, black, near the (A) silver exposed face at 270K, (B) 285K and (C) iodide exposed face at 270K.]{\label{betaprofile}\includegraphics[height = 6cm]{betagraph.pdf}}
	\end{center}
	\label{beta0001}
	\caption{First layers of water on the $\beta$-AgI(0001) face.}
\end{figure}

\indent Both the silver and iodide exposed $\beta$-AgI(0001) faces organize the first layer of water into hexagonal rings, Fig.~\ref{betaface}.  However, the rings formed by each face are different, with the silver exposed face forming rings in the chair conformation and the iodide face forming rings that are planar.  This difference is caused by the ability of the silver exposed face to accept bonding from a hydrogen of water while the iodide exposed face can not.  In Fig.~\ref{betaface} some of the waters on the silver exposed face can be seen to be oriented with one hydrogen obscured below the oxygen as in is pointed toward the surface.  By bonding to the surface, a water molecule will then sit closer to the silver iodide surface than the water molecules around it.  This separation allows the creation of the chair conformation of the hexagonal rings that make up ice.  This bonding may be caused by the coulombic attraction between the positive hydrogen and the negative iodide ions below the silver ions of the silver exposed face.  The iodide face, in contrast, shows no hydrogen pointing toward the surface, as hydrogen is now repulsed by the positive silver ions below the surface iodides.	\\
\indent Density profiles in Fig.~\ref{betaprofile} agree with the assessment of hydrogen bonding to the silver exposed face and not bonding to the iodide exposed face.  Profile A shows the density of ice near the silver iodide surface.  The oxygen density shows a repeating doublet from each bilayer of ice.  Hydrogen is arranged in a triplet around the doublet with the middle peak arising from hydrogen bonding within the bilayer and the outer peaks from bonding to neighbouring bilayers or silver iodide.  Profile B shows the pre-freezing density profile of water on the silver exposed face and it is clear that the first layer of ice is already partially formed by interactions with the silver iodide surface.  In both profiles A and B, a hydrogen peak is clearly seen directed toward the silver iodide surface, whereas the iodide exposed face shown in profile C shows no peak from hydrogens directed toward the silver iodide surface.  The surface layer of water shows that both hydrogen and oxygen remain largely as a single large peak with some hydrogen density showing bonding to water in the next layer.  Monte Carlo calculations of water on $\beta$-AgI(0001) carried out by Shevkunov agree with our results of hydrogen interacting with the second layer of ion.  He also found that the silver exposed surface has a lower energy than the iodide exposed face when water is on the surface\cite{S14}.


\begin{figure}[htp]
	\begin{center}
		\subfigure[The first layer of water near the silver and iodide exposed faces, left and right respectively.  Yellow highlights the FCC face made by the first layer of water, and light blue the FCC face made by the second layer of water.]{\label{gammaface}\includegraphics[height = 6cm]{gammaface.pdf}}
		\subfigure[Normalized density profile of oxygen, red, and hydrogen, black, near the (A) silver exposed face at 270K, (B) 300K and (C) iodide exposed face at 270K.]{\label{gammaprofile}\includegraphics[height = 6cm]{gammagraph.pdf}}
	\caption{First two layers of water on the $\gamma$-AgI(001) face.}
	\end{center}
	\label{gamma001}
\end{figure}

\indent Although a different face, the $\gamma$-AgI(001) face has very similar behaviour to the $\beta$-AgI(0001) face with respect to its ice nucleating ability of the silver and iodide exposed faces.  The faces are cubic, Fig.~\ref{gammaface}, and arrange water into a FCC pattern as is highlighted by the yellow and light blue lines and dots on the figures.  Only the closest layer of water is organized on the iodide exposed face, while the organization extends beyond the first layer on the silver exposed face.  Again, hydrogen can be seen bonding to the silver exposed face, but not to the iodide exposed face, as it may be attracted to iodide ions below the silver ions.  Density profiles in Fig.~\ref{gammaprofile} show the density of water near the $\gamma$-AgI surface.  Profile A is after the water has frozen into cubic ice.  Oxygen gives a repeating peak from the the FCC lattice and half the tetrahedral holes occupied by water.  Hydrogen shows they are oriented to bond to water molecules in neighbouring layers, or to the silver iodide crystal.  Before freezing, Profile B, water shows a similar density profile to ice in its first two layers as the surface organizes the water.  This order is lost on the iodide face, Profile C, as the first oxygen peak is not as sharp as on the silver exposed face, and the hydrogen density shows that rather than bond to the crystal, hydrogens are pointing away from it.
\begin{wrapfigure}{tl}{6cm}
	\begin{center}
		\includegraphics[width = 6cm]{1010face.pdf}
		\caption{First layer of water near the $\beta$-AgI(10$\bar{1}$0) face.  Yellow lines show a ring that is concave out of the page, and the light blue a ring that is convex.}
	\label{1010face}
	\end{center}
	
\end{wrapfigure}
\indent The lattice mismatch for the silver and iodide exposed faces are the same, yet by changing which ion is exposed on the surface the ability of the surface to nucleate ice is created or stopped.  The $\gamma$-AgI(111) face, due to its similarity to the $\beta$-AgI(0001) face shows similar behaviour.  These results demonstrate that a low lattice mismatch with ice is insufficient to predict the ice nucleation behaviour of a material.  To nucleate ice, the surface must be able to organize water into an ice-like structure which likely entails the ability to accept bonding from some hydrogens.\\
\indent The seventh face that was simulated was the $\beta$-AgI(10$\bar{1}$0) face, Fig.~\ref{1010face}.  Compared to the $\beta$-AgI(0001) face, which exhibits an array of chair conformed hexagonal rings, $\beta$-AgI(10$\bar{1}$0) exposes a face of hexagonal rings in the boat conformation.  This face corresponds very well to the same face of I$_{h}$ which produces boat shaped rings between each bilayer.  Lattice mismatch between each face is small, however, no ice was seen to nucleate on this face.  We suspect that although the rings are the right size for ice, the fact that they are made of different ions with differing charges and sizes prevents a favourable match in a way that will nucleate ice.\\
\indent Additional simulations have been done with the $\beta$-AgI(0001) face with various changes to the surface and/or to the interaction parameters used .  To test the flexibility of the silver iodide model we varied the magnitude of charge given to each ion from 0e to 1.0e.  Most simulations were run with a charge of 0.6e, however we observed ice nucleation for charges from 0-0.8e, although the 0.8e simulation took 473ns to nucleate.  Simulations with no charge are interesting as they suggest that the atomistic roughness of the surface can be enough to nucleate ice.  We also simulated a 2D surface by bringing the silver and iodide ions into the same plane and varying the Lennard-Jones parameters.  For the usual parameters and when all the ions use the iodide parameters ice still nucleated, however when all the ions used the silver parameters no ice nucleation was observed.
%Ongoing and future work**********
\subsection{Ongoing and future work}
\subsubsection{AgI Disks}
\begin{figure}[htp]
	\begin{center}
		\includegraphics[width=15cm]{disks.pdf}
		\label{disks}
		\caption{Hexagonal disks used in simulations.  From left to right the diameters are 0.69nm, 1.15nm, 1.61nm and 2.07nm.}
	\end{center}
\end{figure}




\indent Having established a mechanism by which ice will and will not nucleate on an infinite silver iodide surface we are turning our attention to the effect of the size of the slab of silver iodide.  Simulations are under way with hexagonal disks showing the $\beta$-AgI(0001) face.  We hope to establish a size and temperature dependence of ice nucleation on this surface and compare those results to classical nucleation theory applied to nucleation of various shapes of nuclei on surfaces of various geometries\cite{F63}.\\
\indent To date simulations have been run on the disks pictured in Fig.~\ref{disks}.  Three of the larger disks have produced stable clusters of ice at supercooling of ~20 degrees, and the larger two disks have nucleated bulk ice when supercooled ~30 degrees with the six-site model, but not with TIP4P/Ice.  Fig.~\ref{diskcluster} shows clusters of ice on the 2.07nm disk.
\begin{wrapfigure}{hltp}{6cm}
	\begin{center}
		\includegraphics[width=6cm]{207diskcluster.pdf}
		\caption{Clusters of ice on two facing $\beta$-AgI(0001) disks.  Silver and iodide ions and hydrogen are silver, green and black respectively.  Oxygen of hexagonal and cubic ice are blue and orange respectively.}
	\end{center}
	\label{diskcluster}
\end{wrapfigure}

\subsubsection{Kaolinite and Feldspar}
\indent Previous work on this project used Grand Canonical Monte Carlo calculations to study water on the surface of the clay kaolinite\cite{CBP08, CBP09, CBP10a, CBP10b}, Al$_{2}$Si$_{2}$O$_{5}$(OH)$_{4}$.  It was found that water on the clay's surface was organized into hexagonal rings, but the structure did not match that of ice.  Surface defects in he form of trenches were also studied and it was found that the water was more ordered in the trenches due to electric fields generated in them.  Recently, molecular dynamics simulations on kaolinite have shown that ice nucleating on the surface is completely I$_{h}$ and has the bilayer oriented perpendicular to the kaolinite surface\cite{CRKSM13}, and we have observed similar behaviour in a recent simulation.\\
\indent Thus far we have carried out simulations on two (001) faces of fixed kaolinite and observed no ice nucleation.  The mentioned study\cite{CRKSM13} used free kaolinite in the simulations and noted that 1/3 of the surface hydroxides where aligned parallel to the crystal face and could thus accept hydrogen bonds from water.  We are now running simulations with surface hydroxides that are free to rotate.\\
\indent Recent experimental studies on ice nucleation by clays has revealed that the feldspar family of clays is the dominant IN of airborne clay particles\cite{A13}.  Feldspar has three end members that differ in their content of potassium, sodium and calcium: orthoclase KAlSi$_{3}$O$_{8}$, albite NaAlSi$_{3}$O$_{8}$ and anorthite CaAl$_{2}$Si$_{2}$O$_{8}$.  Of these three end members, orthoclase was found to be the most prolific ice nucleus.  We will carry out simulations of water on these clays to determine the reason for their effectiveness in ice nucleation and the differences between them.
\subsubsection{Graphite}
	Recently a simulation showing ice formation on flat and curved sheets of graphene was reported\cite{LHM14}, with decreased nucleation ability of the sheet as it was bent.  These results are interesting, however we feel that the mW model of water used in the simulation may be influencing the results.  The mW model is a coarse grained model of water that represents water as a single point, using a 2-body potential to regulate distances between waters and an 3-body potential to order water into a tetrahedral geometry.  Bulk properties of water are represented well by this model, however it does nucleate ice very readily which may be over emphasizing the ice nucleation ability of graphene.  Also the lack of hydrogen may be affecting the results, as we have observed for silver iodide, hydrogen plays a crucial role in ice nucleation and not representing hydrogen may affect the results.  We will explore the ice nucleating ability of graphene using the atomistic models that have been mentioned.

\subsubsection{Pseudomonas syringae}
Proteins have been observed to both nucleate ice and inhibit ice growth.  Among ice nucleation proteins, one of the most effective is produced by the bacteria {\it Pseudomonas syringae}.  The gene that produces this protein has been sequenced\cite{GW85} and the corresponding protein contains 1200 amino acids with 976 of the amino acids from the mid-part of the protein showing a repeating pattern every 8, 16 and 48 amino acids.  A simulation on part of the repeating section of the ice nucleation protein from {\it Pseudomonas borealis}, which is very similar to the protein from {\it Pseudomonas syringae} predicted a $\beta$-helix structure of that region\cite{GCWD11}.   This section localized water on its surface in spacings similar to that of I$_{h}$.  They also found that the $\beta$-helix could dimerize and form a larger nucleation surface.  We will try to perform similar simulations with the protein from {\it Pseudomonas syringae} and cool the simulation in an attempt to nucleate ice.

\section{Conclusion}

\singlespacing
\bibliographystyle{unsrt}
\bibliography{references}


\end{document}