\documentclass[titlepage,11pt]{article}

\usepackage{amsmath, amssymb}
\usepackage{float}
\usepackage{gensymb}
\usepackage[pdftex]{graphicx}
\usepackage[margin=1in]{geometry}
\usepackage{ragged2e}
\usepackage{setspace}
\usepackage{cleveref}
\usepackage{siunitx}% si units
\usepackage{wrapfig}
\usepackage[utf8]{inputenc} % Required for inputting international characters
\usepackage[T1]{fontenc} % Output font encoding for international characters
\usepackage{mathpazo} % Palatino font
\usepackage{xcolor}
\usepackage[sort&compress,numbers,super]{natbib}
\usepackage{caption}
\usepackage{subcaption}
\usepackage{chngpage}

\newcommand{\figfile}{./figures}
\newcommand{\HOS}{{\color{red}HOS}}
\newcommand{\Ham}{\widehat{\mathcal{H}}}

\begin{document}
%----------------------------------------------------------------------------------------
%	TITLE PAGE
%----------------------------------------------------------------------------------------

\begin{titlepage} % Suppresses displaying the page number on the title page and the subsequent page counts as page 1
	\newcommand{\HRule}{\rule{\linewidth}{0.5mm}} % Defines a new command for horizontal lines, change thickness here
	
	\center % Centre everything on the page
	\begin{figure*}
		\centering
		\includegraphics[width=\textwidth]{figures/UBC_Chemlogo.eps}
	\end{figure*}
	
	
	%------------------------------------------------
	%	Headings
	%------------------------------------------------
	
	\textsc{\LARGE Fourth Year Meeting\\ \vspace{1em}  \& Thesis Outline}\\[1.5cm] % Main heading such as the name of your university/college
	
	
	%------------------------------------------------
	%	Title
	%------------------------------------------------
	
	\HRule\\[0.4cm]
	
	{\huge\bfseries Exploration of Crystal Nucleation Phenomena Through Molecular Simulation}\\[0.4cm] % Title of your document
	
	\HRule\\[1.5cm]
	
	%------------------------------------------------
	%	Author(s)
	%------------------------------------------------
	
	\begin{minipage}{0.4\textwidth}
		\begin{flushleft}
			\large
			\textit{Author}\\
			Hayden \textsc{Scheiber} % Your name
		\end{flushleft}
	\end{minipage}
	~
	\begin{minipage}{0.4\textwidth}
		\begin{flushright}
			\large
			\textit{Supervisor}\\
			Dr. Gren \textsc{Patey} % Supervisor's name
		\end{flushright}
	\end{minipage}\\

	\vspace{1cm}
	
	%\begin{minipage}{0.4\textwidth}
	%	\begin{flushleft}
	%		\large
	%		\textit{Chair}\\
	%		Dr. Geoffrey \textsc{Herring} % Chair
	%	\end{flushleft}
	%\end{minipage}
	~
	\begin{minipage}{0.4\textwidth}
		\begin{center}
			\large
			\textit{Committee}\\
			Dr. Yan \textsc{Wang}\\ % Committee
			Dr. Mark \textsc{Thachuk}\\ % Committee
			Dr. Keng \textsc{Chou} % Committee
		\end{center}
	\end{minipage}
	
	%------------------------------------------------
	%	Date
	%------------------------------------------------
	
	\vfill\vfill\vfill % Position the date 3/4 down the remaining page
	
	{\large
	\begin{table}[H]
		\centering
		\begin{tabular}{lll}
			\textbf{Date} & \textbf{Time} & \textbf{Location} \\
			Monday, October 4$^{\text{th}}$ 2021 & 11:00 AM & CHEM D317
		\end{tabular}
	\end{table}
	}
	%------------------------------------------------
	%	Logo
	%------------------------------------------------
	
	%\vfill\vfill
	%\includegraphics[width=0.2\textwidth]{placeholder.jpg}\\[1cm] % Include a department/university logo - this will require the graphicx package
	 
	%----------------------------------------------------------------------------------------
	
	\vfill % Push the date up 1/4 of the remaining page
	
\end{titlepage}

%----------------------------------------------------------------------------------------

%\large
\justifying

\section{Analysis of the relative stability of lithium halide crystal structures: Density functional theory and classical models}

\begin{itemize}
	\item This project is a explores the relative energetics of low-lying lithium halide crystal structures using both density functional theory and pairwise classical models.
	\item There is a large amount of work associated with this project, I expect it will occupy approximately 1/4 to 1/3 of the thesis.
	\item \textbf{Status: 100 \% complete.} This project is published.\footnote{Scheiber, H. O., and G. N. Patey. The Journal of Chemical Physics 154.18 (\textbf{2021}): 184507}.
\end{itemize}


\section{Bayesian Optimization for Lithium Halide Forcefields: Exploring the Limits of Pairwise Additive Potentials for Molecular Simulation}

\begin{itemize}
	\item This project employs a machine learning technique, Bayesian optimization, for finding pairwise lithium halide forcefields that reproduce known experimental or theoretical quantities.
	\item \textbf{Status: 90 \% complete.} All relevant project code, most data collection, and most data analysis is complete.
\end{itemize}


\section{The NiAs Crystal Structure of LiI: Low Energy, but Never Observed}

\begin{itemize}
	\item This project combines aspects of both chapter 1 and 2, employing the high-quality reference data we produced in chapter 1 with the methodology from chapter 2 to explore why the \mbox{($T$ = 0, $P$ = 0)} energy of LiI in the NiAs structure is lower than LiI in the rocksalt structure, yet NiAs is not observed experimentally.
	\item \textbf{Status: 70 \% complete.} Relevant project code is complete and much of the data is collected. A few more simulations are required.
\end{itemize}

\section{Convolutional Neural Networks with Steinhart Order Parameters: Machine Learning for Structure Detection in Molecular Simulation}

\begin{itemize}
	\item This project came out of the need to distinguish different lithium halide crystal structures in simulation. It employs neural networks for local crystal structure detection.
	\item \textbf{Status: 80 \% complete.} Relevant project code is complete. Still need to decide on an interesting simulation to showcase this project, then write it up.
\end{itemize}

\section{Competitive Nucleation of LiX Crystal Structures}

\begin{itemize}
	\item This project is still in progress, but will use aspects of several previous chapters to explore competitive nucleation in LiX crystal structures.
	\item \textbf{Status: 20 \% complete.} 
\end{itemize}

\textbf{Expected date of program completion: September 2022.}



%\singlespacing
%\setlength{\bibsep}{0pt plus 0.0ex}
%\bibliographystyle{achemso}
%\bibliography{references}

\end{document}